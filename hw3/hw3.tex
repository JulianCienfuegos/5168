\documentclass[10pt]{article}

\usepackage[margin=0.75in]{geometry}
\usepackage{fancyhdr}
\pagestyle{fancy}
\usepackage{amsmath}
\usepackage{amssymb}
\usepackage{color}

\definecolor{mygreen}{rgb}{0,0.6,0}
\definecolor{mygray}{rgb}{0.5,0.5,0.5}
\definecolor{mymauve}{rgb}{0.58,0,0.82}
\definecolor{mylilas}{RGB}{170,55,241}
\usepackage{graphicx}
\usepackage{listings}
\lstset{language=Matlab,%
    %basicstyle=\color{red},
    breaklines=true,%
    morekeywords={matlab2tikz},
    keywordstyle=\color{blue},%
    morekeywords=[2]{1}, keywordstyle=[2]{\color{black}},
    identifierstyle=\color{black},%
    stringstyle=\color{mylilas},
    commentstyle=\color{mygreen},%
    showstringspaces=false,%without this there will be a symbol in the places where there is a space
    numbers=left,%
    flexiblecolumns=true,
    numbersep=9pt, % this defines how far the numbers are from the text
    emph=[1]{for,end,break},emphstyle=[1]\color{red}, %some words to emphasise
    %emph=[2]{word1,word2}, emphstyle=[2]{style}, 
    stepnumber=1  
}
\allowdisplaybreaks

\lhead{FEM 5168 HW3}
\chead{Melvyn Ian Drag}
\rhead{\today}
\setlength{\parskip}{0pt} 
\setlength{\parindent}{0pt}
\newcommand{\tab}[1]{\hspace*{4ex}\rlap{#1}}
\newcommand{\tbf}[1]{\textbf{#1}}
\newcommand{\ptl}[2]{\frac{\partial^2 #1}{\partial #2 ^2}}
\newcommand{\der}[2]{\frac{d #1}{d #2}}
\newcommand{\iab}[2]{\int_{ #1 }^{ #2 }}

\begin{document}
\section*{Parts a - c}
\subsection*{Element1D}
\begin{lstlisting}
function [ke, fe] = element1d(psi)
    %{
        The shape functions will have degree d = length(psi) - 1.
        Therefore the product of two of these functions will have degree
        2*d.
        The derivatives of the shape functions will have degree d - 1. 
        Therefore the product of any two of these functions will have
        degree 2*d - 2. 
    %}

    d = length(psi) - 1;
    % Different integrals need different numbers of points.
    n_kk = ceil(((2*d-2) + 1)/2);
    [x_gauss_kk, w_gauss_kk] = gauss_points_and_weights(n_kk);

    n_kb = ceil((2*d+1)/2);
    [x_gauss_kb, w_gauss_kb] = gauss_points_and_weights(n_kb);

    n_f = ceil((d+1)/2);
    [x_gauss_f, w_gauss_f] = gauss_points_and_weights(n_f);
    % memory allocation
    ke.k = zeros(d+1);
    ke.b = zeros(d+1);
    fe = zeros(d+1, 1);

    for i = 1:d+1
        for j = 1:i
            % make the components of ke
            g_kk = conv(psi(i).der, psi(j).der);
            g_gauss_kk = polyval(g_kk, x_gauss_kk);
            ke.k(i, j) = dot(w_gauss_kk, g_gauss_kk);

            g_kb = conv(psi(i).fun, psi(j).fun);
            g_gauss_kb = polyval(g_kb, x_gauss_kb);
            ke.b(i, j) = dot(w_gauss_kb, g_gauss_kb);
            % enforce symmetry
            ke.k(j, i) = ke.k(i, j);
            ke.b(j, i) = ke.b(i, j);
        end
        % make the componenets of fe.
        f_gauss = polyval(psi(i).fun, x_gauss_f);
        fe(i) = dot(w_gauss_f, f_gauss);
    end
end
\end{lstlisting}
\subsection*{Global Assembly}
\begin{lstlisting}
function [K,F] = assemble1D(k, f, K, F, node_list)
    s = node_list(1);
    t = node_list(2);
    K(s:t, s:t) = K(s:t, s:t) + k;
    F(s:t) = F(s:t) + f;
end
\end{lstlisting}
\subsection*{Boundary Conditions}
\begin{lstlisting}
function [K, F] = enforce_boundaries(K, F, NODEBC1, NODEBC2, VBC1, VBC2, KofX)
    for i = 1:length(NODEBC1)
        idx = NODEBC1(i);             % Where is the dirichlet condition?
        if(idx~=0)		
            F = F - VBC1(i)*K(:,idx); % Modify F
            F(idx) = VBC1(i);
            K(idx,:) = 0;             % Zero out the row and column.
            K(:,idx) = 0;
            K(idx, idx) = 1;          % Set diagonal entry to one.
        end
    end
    for i = 1:length(NODEBC2)
        idx = NODEBC2(i);             % where is the neumann condition?
        if (idx == 1)                 % modify f
            F(idx) = F(idx) - KofX(1)*VBC2(i);
        else if (idx > 1)
            F(idx) = F(idx) + KofX(end)*VBC2(i);
        end        
    end
end
\end{lstlisting}
\subsection*{One Script to Rule Them All}
\begin{lstlisting}
%{
    Solve a problem using the FEM.
%}
read_1D_mesh();
read_1D_input();
K = zeros(nnodes);
F = zeros(nnodes, 1);
psi = lagrange_poly(p); % p is assigned in read_1d_mesh
[ke, fe] = element1d(psi);
for n = 1:nelems
    node_list = CONN(n,:);
    h = XNODES(node_list(2)) - XNODES(node_list(1)); 
    % Contributions to the global matrix
    k = (2/h)*KofX(n)*ke.k + (h/2)*BofX(n)*ke.b;
    f_c = (h/2)*FofX(n)*fe;
    % Now plug these contributions into the appropriate place.
    [K, F] = assemble1D(k, f_c, K, F, node_list);
end
[K, F] = enforce_boundaries(K, F, NODEBC1, NODEBC2, VBC1, VBC2, KofX);

% Calculate a solution
u = K\F;

% Compare the solution to the real thing. View the plot and the error.
d = length(u);
u_exact = @(x) log(1+x)/log(2) - x;
x = linspace(0, 1, d);
figure()
plot(x, u_exact(x), 'b', x, u, 'r')
legend('y = u\_exact(x)','y = u\_approx','Location','northeast')
title('Comparison of FEM Solution to Exact Solution.')
x1 = 0.7;
y1 = 0.075;
str1 = strcat('Norm of error: ', num2str(norm(u' - u_exact(x), inf)));
text(x1, y1, str1);
xlabel('x')
ylabel('u(x)')
\end{lstlisting}
\section*{Part d}
\subsection*{Test Problem 1}
\lstset{stepnumber=0}
\begin{lstlisting}

K =

    1.0000         0         0         0         0         0
         0   30.0000  -25.5000         0         0         0
         0  -25.5000   32.5000   -7.0000         0         0
         0         0   -7.0000   22.5000  -15.5000         0
         0         0         0  -15.5000   20.0000         0
         0         0         0         0         0    1.0000


F =

         0
    0.1500
    0.1250
    0.1500
    0.2500
         0
\end{lstlisting}
\subsection*{Test Problem 2}
\begin{lstlisting}
K =

    1.0000         0         0         0         0
         0    8.1667   -3.9583         0         0
         0   -3.9583    8.1667   -3.9583         0
         0         0   -3.9583    8.1667   -3.9583
         0         0         0   -3.9583    4.0833


F =

    2.0000
    8.4792
    0.6250
    0.6875
    2.3594
\end{lstlisting}
\end{document}