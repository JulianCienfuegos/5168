\documentclass[10pt]{article}

\usepackage[margin=1in]{geometry}
\usepackage{fancyhdr}
\pagestyle{fancy}
\usepackage{amsmath}
\usepackage{amssymb}
\usepackage{color}

\definecolor{mygreen}{rgb}{0,0.6,0}
\definecolor{mygray}{rgb}{0.5,0.5,0.5}
\definecolor{mymauve}{rgb}{0.58,0,0.82}
\definecolor{mylilas}{RGB}{170,55,241}
\usepackage{graphicx}
\usepackage{listings}
\lstset{language=Matlab,%
    %basicstyle=\color{red},
    frame=single, 
    breaklines=true,%
    morekeywords={matlab2tikz},
    keywordstyle=\color{blue},%
    morekeywords=[2]{1}, keywordstyle=[2]{\color{black}},
    identifierstyle=\color{black},%
    stringstyle=\color{mylilas},
    commentstyle=\color{mygreen},%
    showstringspaces=false,%without this there will be a symbol in the places where there is a space
    numbers=left,%
    flexiblecolumns=true,
    numbersep=9pt, % this defines how far the numbers are from the text
    emph=[1]{for,end,break},emphstyle=[1]\color{red}, %some words to emphasise
    %emph=[2]{word1,word2}, emphstyle=[2]{style}, 
    stepnumber=1  
}
\allowdisplaybreaks

\lhead{\large 5168 hw4 part duex}
\chead{\large Melvyn Ian Drag}
\rhead{\large\today}
\setlength{\parskip}{0pt} 
\setlength{\parindent}{0pt}
\newcommand{\tab}[1]{\hspace*{4ex}\rlap{#1}}
\newcommand{\tbf}[1]{\textbf{#1}}
\newcommand{\ptl}[2]{\frac{\partial^2 #1}{\partial #2 ^2}}
\newcommand{\der}[2]{\frac{d #1}{d #2}}
\newcommand{\iab}[2]{\int_{ #1 }^{ #2 }}

\begin{document}
\section*{Problem 2}
The \textbf{X} and \textbf{Y} computations are identical, so I will just do the \textbf{X} computation and the result will be the same for \textbf{Y} with the variables swapped. Remember $\int dA = \frac12$. 

\begin{gather*}
k_{11x}^e = \frac12*\frac{k_0}{2A_e}\left[((x_3 - x_1)(-1) - (x_2 - x_1)(-1))((x_3 - x_1)(-1) - (x_2 - x_1)(-1))\right]\\
=\frac{k_0}{4A_e}(x_2 - x_3)^2\\
k_{12x}^e =\frac12*\frac{k_0}{2A_e}\left[((x_3 - x_1)(-1) - (x_2 - x_1)(-1))((x_3 - x_1)(1) - (x_2 - x_1)(0))\right]\\
=\frac12*\frac{k_0}{2A_e}\left[(x_1 - x_3 + x_2 - x_1)(x_3 - x_1)\right]\\
=\frac12*\frac{k_0}{2A_e}(x_2-x_3)(x_3-x_1)\\
 k_{13x}^e =\frac12*\frac{k_0}{2A_e}\left[((x_3 - x_1)(-1) - (x_2 - x_1)(-1))((x_3 - x_1)(0) - (x_2 - x_1)(1))\right]\\
=\frac12*\frac{k_0}{2A_e}(-x_3+x_1+x_2-x_1)(x_1-x_2)\\
=\frac12*\frac{k_0}{2A_e}[(-x_3+x_2)(x_1-x_2)]\\
 k_{22x}^e =\frac12*\frac{k_0}{2A_e}\left[((x_3 - x_1)(1) - (x_2 - x_1)(0))((x_3 - x_1)(1) - (x_2 - x_1)(0))\right]\\
=\frac12*\frac{k_0}{2A_e}(x_3-x_1)^2\\
k_{23x}^e =\frac12*\frac{k_0}{2A_e}\left[((x_3 - x_1)(1) - (x_2 - x_1)(0))((x_3 - x_1)(0) - (x_2 - x_1)(1))\right]\\
=\frac12*\frac{k_0}{2A_e}(x_3 - x_1)(x_1-x_2)\\
k_{33x}^e =\frac12*\frac{k_0}{2A_e}\left[((x_3 - x_1)(0) - (x_2 - x_1)(1))((x_3 - x_1)(0) - (x_2 - x_1)(1))\right]\\
=\frac12*\frac{k_0}{2A_e}(x_1-x_2)^2\\
\end{gather*}
The sub-diagonal elements can be filled with their corresponding symmetric values. Summing the \textbf{X} and \textbf{Y} contributions gives the result.

\end{document}